% Options for packages loaded elsewhere
\PassOptionsToPackage{unicode}{hyperref}
\PassOptionsToPackage{hyphens}{url}
%
\documentclass[
]{book}
\usepackage{amsmath,amssymb}
\usepackage{lmodern}
\usepackage{iftex}
\ifPDFTeX
  \usepackage[T1]{fontenc}
  \usepackage[utf8]{inputenc}
  \usepackage{textcomp} % provide euro and other symbols
\else % if luatex or xetex
  \usepackage{unicode-math}
  \defaultfontfeatures{Scale=MatchLowercase}
  \defaultfontfeatures[\rmfamily]{Ligatures=TeX,Scale=1}
\fi
% Use upquote if available, for straight quotes in verbatim environments
\IfFileExists{upquote.sty}{\usepackage{upquote}}{}
\IfFileExists{microtype.sty}{% use microtype if available
  \usepackage[]{microtype}
  \UseMicrotypeSet[protrusion]{basicmath} % disable protrusion for tt fonts
}{}
\makeatletter
\@ifundefined{KOMAClassName}{% if non-KOMA class
  \IfFileExists{parskip.sty}{%
    \usepackage{parskip}
  }{% else
    \setlength{\parindent}{0pt}
    \setlength{\parskip}{6pt plus 2pt minus 1pt}}
}{% if KOMA class
  \KOMAoptions{parskip=half}}
\makeatother
\usepackage{xcolor}
\IfFileExists{xurl.sty}{\usepackage{xurl}}{} % add URL line breaks if available
\IfFileExists{bookmark.sty}{\usepackage{bookmark}}{\usepackage{hyperref}}
\hypersetup{
  pdftitle={Data Science 1},
  pdfauthor={Sebastian Sauer},
  hidelinks,
  pdfcreator={LaTeX via pandoc}}
\urlstyle{same} % disable monospaced font for URLs
\usepackage{color}
\usepackage{fancyvrb}
\newcommand{\VerbBar}{|}
\newcommand{\VERB}{\Verb[commandchars=\\\{\}]}
\DefineVerbatimEnvironment{Highlighting}{Verbatim}{commandchars=\\\{\}}
% Add ',fontsize=\small' for more characters per line
\usepackage{framed}
\definecolor{shadecolor}{RGB}{248,248,248}
\newenvironment{Shaded}{\begin{snugshade}}{\end{snugshade}}
\newcommand{\AlertTok}[1]{\textcolor[rgb]{0.94,0.16,0.16}{#1}}
\newcommand{\AnnotationTok}[1]{\textcolor[rgb]{0.56,0.35,0.01}{\textbf{\textit{#1}}}}
\newcommand{\AttributeTok}[1]{\textcolor[rgb]{0.77,0.63,0.00}{#1}}
\newcommand{\BaseNTok}[1]{\textcolor[rgb]{0.00,0.00,0.81}{#1}}
\newcommand{\BuiltInTok}[1]{#1}
\newcommand{\CharTok}[1]{\textcolor[rgb]{0.31,0.60,0.02}{#1}}
\newcommand{\CommentTok}[1]{\textcolor[rgb]{0.56,0.35,0.01}{\textit{#1}}}
\newcommand{\CommentVarTok}[1]{\textcolor[rgb]{0.56,0.35,0.01}{\textbf{\textit{#1}}}}
\newcommand{\ConstantTok}[1]{\textcolor[rgb]{0.00,0.00,0.00}{#1}}
\newcommand{\ControlFlowTok}[1]{\textcolor[rgb]{0.13,0.29,0.53}{\textbf{#1}}}
\newcommand{\DataTypeTok}[1]{\textcolor[rgb]{0.13,0.29,0.53}{#1}}
\newcommand{\DecValTok}[1]{\textcolor[rgb]{0.00,0.00,0.81}{#1}}
\newcommand{\DocumentationTok}[1]{\textcolor[rgb]{0.56,0.35,0.01}{\textbf{\textit{#1}}}}
\newcommand{\ErrorTok}[1]{\textcolor[rgb]{0.64,0.00,0.00}{\textbf{#1}}}
\newcommand{\ExtensionTok}[1]{#1}
\newcommand{\FloatTok}[1]{\textcolor[rgb]{0.00,0.00,0.81}{#1}}
\newcommand{\FunctionTok}[1]{\textcolor[rgb]{0.00,0.00,0.00}{#1}}
\newcommand{\ImportTok}[1]{#1}
\newcommand{\InformationTok}[1]{\textcolor[rgb]{0.56,0.35,0.01}{\textbf{\textit{#1}}}}
\newcommand{\KeywordTok}[1]{\textcolor[rgb]{0.13,0.29,0.53}{\textbf{#1}}}
\newcommand{\NormalTok}[1]{#1}
\newcommand{\OperatorTok}[1]{\textcolor[rgb]{0.81,0.36,0.00}{\textbf{#1}}}
\newcommand{\OtherTok}[1]{\textcolor[rgb]{0.56,0.35,0.01}{#1}}
\newcommand{\PreprocessorTok}[1]{\textcolor[rgb]{0.56,0.35,0.01}{\textit{#1}}}
\newcommand{\RegionMarkerTok}[1]{#1}
\newcommand{\SpecialCharTok}[1]{\textcolor[rgb]{0.00,0.00,0.00}{#1}}
\newcommand{\SpecialStringTok}[1]{\textcolor[rgb]{0.31,0.60,0.02}{#1}}
\newcommand{\StringTok}[1]{\textcolor[rgb]{0.31,0.60,0.02}{#1}}
\newcommand{\VariableTok}[1]{\textcolor[rgb]{0.00,0.00,0.00}{#1}}
\newcommand{\VerbatimStringTok}[1]{\textcolor[rgb]{0.31,0.60,0.02}{#1}}
\newcommand{\WarningTok}[1]{\textcolor[rgb]{0.56,0.35,0.01}{\textbf{\textit{#1}}}}
\usepackage{longtable,booktabs,array}
\usepackage{calc} % for calculating minipage widths
% Correct order of tables after \paragraph or \subparagraph
\usepackage{etoolbox}
\makeatletter
\patchcmd\longtable{\par}{\if@noskipsec\mbox{}\fi\par}{}{}
\makeatother
% Allow footnotes in longtable head/foot
\IfFileExists{footnotehyper.sty}{\usepackage{footnotehyper}}{\usepackage{footnote}}
\makesavenoteenv{longtable}
\usepackage{graphicx}
\makeatletter
\def\maxwidth{\ifdim\Gin@nat@width>\linewidth\linewidth\else\Gin@nat@width\fi}
\def\maxheight{\ifdim\Gin@nat@height>\textheight\textheight\else\Gin@nat@height\fi}
\makeatother
% Scale images if necessary, so that they will not overflow the page
% margins by default, and it is still possible to overwrite the defaults
% using explicit options in \includegraphics[width, height, ...]{}
\setkeys{Gin}{width=\maxwidth,height=\maxheight,keepaspectratio}
% Set default figure placement to htbp
\makeatletter
\def\fps@figure{htbp}
\makeatother
\setlength{\emergencystretch}{3em} % prevent overfull lines
\providecommand{\tightlist}{%
  \setlength{\itemsep}{0pt}\setlength{\parskip}{0pt}}
\setcounter{secnumdepth}{5}
\usepackage{booktabs}
\usepackage{amsthm}
\makeatletter
\def\thm@space@setup{%
  \thm@preskip=8pt plus 2pt minus 4pt
  \thm@postskip=\thm@preskip
}
\makeatother
\ifLuaTeX
  \usepackage{selnolig}  % disable illegal ligatures
\fi
\usepackage[]{natbib}
\bibliographystyle{apalike}

\title{Data Science 1}
\author{Sebastian Sauer}
\date{2022-02-21 21:56:46}

\begin{document}
\maketitle

{
\setcounter{tocdepth}{1}
\tableofcontents
}
\begin{Shaded}
\begin{Highlighting}[]
\FunctionTok{library}\NormalTok{(tidyverse)}
\end{Highlighting}
\end{Shaded}

\begin{verbatim}
## -- Attaching packages --------------------------------------- tidyverse 1.3.1 --
\end{verbatim}

\begin{verbatim}
## v ggplot2 3.3.5     v purrr   0.3.4
## v tibble  3.1.6     v dplyr   1.0.8
## v tidyr   1.2.0     v stringr 1.4.0
## v readr   2.1.2     v forcats 0.5.1
\end{verbatim}

\begin{verbatim}
## -- Conflicts ------------------------------------------ tidyverse_conflicts() --
## x dplyr::filter() masks stats::filter()
## x dplyr::lag()    masks stats::lag()
\end{verbatim}

\hypertarget{uxfcberblick}{%
\chapter{Überblick}\label{uxfcberblick}}

\hypertarget{was-sie-hier-lernen-und-wozu-das-gut-ist}{%
\section{Was Sie hier lernen und wozu das gut ist}\label{was-sie-hier-lernen-und-wozu-das-gut-ist}}

Alle Welt spricht von Big Data, aber ohne die Analyse sind die großen Daten nur großes Rauschen. Was letztlich interessiert, sind die Erkenntnisse, die Einblicke, nicht die Daten an sich.
Dabei ist es egal, ob die Daten groß oder klein sind.
Natürlich erlauben die heutigen Datenmengen im Verbund mit leistungsfähigen Rechnern und neuen Analysemethoden ein Verständnis,
das vor Kurzem noch nicht möglich war.
Und wir stehen erst am Anfang dieser Entwicklung.
Vielleicht handelt es sich bei diesem Feld um eines der dynamischsten Fachgebiete der heutigen Zeit.
Sie sind dabei: Sie lernen einiges Handwerkszeugs des ``Datenwissenschaftlers''.
Wir konzentrieren uns auf das vielleicht bekannteste Teilgebiet:
Ereignisse vorhersagen auf Basis von hoch strukturierten Daten
und geeigneter Algorithmen und Verfahren.
Nach diesem Kurs sollten Sie in der Lage sein,
typisches Gebabbel des Fachgebiet mit Lässigkeit mitzumachen.
Ach ja, und mit einigem Erfolg Vorhersagemodelle entwickeln.

\hypertarget{lernziele}{%
\section{Lernziele}\label{lernziele}}

Nach diesem Kurs sollten Sie

\begin{itemize}
\tightlist
\item
  grundlegende Konzepte des statistischen Lernens verstehen und mit R anwenden können
\item
  gängige Prognose-Algorithmen kennen, in Grundzügen verstehen und mit R anwenden können
\item
  die Güte und Grenze von Prognosemodellen einschätzen können
\end{itemize}

\hypertarget{voraussetzungen}{%
\section{Voraussetzungen}\label{voraussetzungen}}

Um von diesem Kurs am besten zu profitieren,
sollten Sie folgendes Wissen mitbringen:

\begin{itemize}
\tightlist
\item
  grundlegende Kenntnisse im Umgang mit R, möglichst auch mit dem tidyverse
\item
  grundlegende Kenntnisse der deskriptiven Statistik
\item
  grundlegende Kenntnis der Regressionsanalyse
\end{itemize}

\hypertarget{hinweise-zu-diesem-projekt}{%
\section{Hinweise zu diesem Projekt}\label{hinweise-zu-diesem-projekt}}

\begin{itemize}
\item
  Die URL zu diesem Projekt lautet \textless test.io\textgreater.
\item
  Lesen Sie sich die folgenden Informationen bitte gut durch: \href{https://sebastiansauer.github.io/fopra/Interna/Hinweise.html}{Hinweise}
\item
  Den Quellcode finden Sie \href{https://github.com/sebastiansauer/datascience1}{in diesem Github-Repo}.
\item
  Sie haben Feedback, Fehlerhinweise oder Wünsche zur Weiterentwicklung? Am besten stellen Sie \href{https://github.com/sebastiansauer/datascience1/issues}{hier} einen \emph{Issue} ein.
\item
  Dieses Projekt steht unter der \href{https://github.com/sebastiansauer/datascience1/blob/main/LICENSE}{MIT-Lizenz}).
\end{itemize}

\hypertarget{lernhilfen}{%
\section{Lernhilfen}\label{lernhilfen}}

\hypertarget{software}{%
\subsection{Software}\label{software}}

\begin{itemize}
\tightlist
\item
  Installieren Sie \href{https://data-se.netlify.app/2021/11/30/installation-von-r-und-seiner-freunde/}{R und seine Freunde}.
\item
  Installieren Sie die folgende R-Pakete:

  \begin{itemize}
  \tightlist
  \item
    tidyverse
  \item
    tidymodels
  \item
    weitere Pakete werden im Unterricht bekannt gegeben (es schadet aber nichts, jetzt schon Pakete nach eigenem Ermessen zu installieren)
  \end{itemize}
\item
  \href{https://github.com/sebastiansauer/Lehre}{R Syntax aus dem Unterricht} findet sich im Github-Repo bzw. Ordner zum jeweiligen Semester.
\end{itemize}

\hypertarget{online-zusammenarbeit}{%
\subsection{Online-Zusammenarbeit}\label{online-zusammenarbeit}}

\begin{itemize}
\tightlist
\item
  \href{https://frag.jetzt/home}{Frag-Jetzt-Raum zum anonymen Fragen stellen während des Unterrichts}. Der Keycode wird Ihnen vom Dozenten bereitgestellt.
\item
  \href{https://de.padlet.com/}{Padlet} zum einfachen (und anonymen) Hochladen von Arbeitsergebnissen der Studentis im Unterricht. Wir nutzen es als eine Art Pinwand zum Sammeln von Arbeitsbeiträgen. Die Zugangsdaten stellt Ihnen der Dozent bereit.
\end{itemize}

\hypertarget{literatur}{%
\section{Literatur}\label{literatur}}

Zentrale Kursliteratur für die theoretischen Konzepte ist \citep{rhys_machine_2020}.
Die praktische Umsetzung in R basiert auf \citep{silge_tidy_nodate}.

\hypertarget{moduluxfcberblick}{%
\chapter{Modulüberblick}\label{moduluxfcberblick}}

\hypertarget{grundkonzepte}{%
\section{Grundkonzepte}\label{grundkonzepte}}

\hypertarget{datum}{%
\subsection{Datum}\label{datum}}

\begin{itemize}
\tightlist
\item
  14.-18.3.22
\end{itemize}

\hypertarget{lernziele-1}{%
\subsection{Lernziele}\label{lernziele-1}}

\begin{itemize}
\tightlist
\item
  Sie können erläutern, was man unter statistischem Lernen versteht.
\item
  Sie wissen, war Overfitting ist, wie es entsteht, und wie es vermieden werden kann.
\item
  Sie kennen verschiedenen Arten von statistischem Lernen und können Algorithmen zu diesen Arten zuordnen.
\end{itemize}

\hypertarget{vorbereitung}{%
\subsection{Vorbereitung}\label{vorbereitung}}

\begin{itemize}
\tightlist
\item
  Lesen Sie die Hinweise zum Modul.
\item
  Installieren (oder Updaten) Sie die für dieses Modul angegeben Software.
\item
  Lesen Sie die Literatur.
\end{itemize}

\hypertarget{literatur-1}{%
\subsection{Literatur}\label{literatur-1}}

\begin{itemize}
\tightlist
\item
  Rhys, Kap. 1
\end{itemize}

\hypertarget{skript}{%
\subsection{Skript}\label{skript}}

\begin{itemize}
\tightlist
\item
  \href{}{Kap. 1}
\end{itemize}

\hypertarget{hinweise}{%
\subsection{Hinweise}\label{hinweise}}

\begin{itemize}
\tightlist
\item
  Bitte beachten Sie die Einteilung in die Züge für den Präsenzunterricht.
\end{itemize}

\hypertarget{tidyverse-2.-blick}{%
\section{tidyverse, 2. Blick}\label{tidyverse-2.-blick}}

\hypertarget{datum-1}{%
\subsection{Datum}\label{datum-1}}

\begin{itemize}
\tightlist
\item
  21.3.-25.3.
\end{itemize}

\hypertarget{lernziele-2}{%
\subsection{Lernziele}\label{lernziele-2}}

\begin{itemize}
\tightlist
\item
  Sie können Funktionen, auch anonyme, in R schreiben.
\item
  Sie können Datensätze vom Lang- und Breit-Format wechseln.
\item
  Sie können Mapping-Funktionen anwenden.
\item
  Sie können eine dplyr-Funktion auf mehrere Spalten gleichzeitig anwenden.
\end{itemize}

\hypertarget{vorbereitung-1}{%
\subsection{Vorbereitung}\label{vorbereitung-1}}

\begin{itemize}
\tightlist
\item
  Lesen Sie die Literatur.
\end{itemize}

\hypertarget{literatur-2}{%
\subsection{Literatur}\label{literatur-2}}

\begin{itemize}
\tightlist
\item
  Rhys, Kap. 2
\end{itemize}

\hypertarget{tidymodels}{%
\section{tidymodels}\label{tidymodels}}

\hypertarget{datum-2}{%
\subsection{Datum}\label{datum-2}}

\begin{itemize}
\tightlist
\item
  28.3.-1.4.
\end{itemize}

\hypertarget{literatur-3}{%
\subsection{Literatur}\label{literatur-3}}

\begin{itemize}
\tightlist
\item
  TMWR
\end{itemize}

\hypertarget{knn}{%
\section{kNN}\label{knn}}

\hypertarget{datum-3}{%
\subsection{Datum}\label{datum-3}}

\begin{itemize}
\tightlist
\item
  4.4.-8.4.
\end{itemize}

\hypertarget{literatur-4}{%
\subsection{Literatur}\label{literatur-4}}

\begin{itemize}
\tightlist
\item
  Rhys, Kap. 3
\end{itemize}

\hypertarget{statistisches-lernen}{%
\section{Statistisches Lernen}\label{statistisches-lernen}}

\hypertarget{datum-4}{%
\subsection{Datum}\label{datum-4}}

\begin{itemize}
\tightlist
\item
  11.4.-15.4.
\end{itemize}

\hypertarget{literatur-5}{%
\subsection{Literatur}\label{literatur-5}}

\begin{itemize}
\tightlist
\item
  Rhys, Kap. 3
\end{itemize}

\hypertarget{hinweise-1}{%
\subsection{Hinweise}\label{hinweise-1}}

\begin{itemize}
\tightlist
\item
  In dieser Woche fällt die Übung aus.
\end{itemize}

\hypertarget{wiederholung}{%
\section{Wiederholung}\label{wiederholung}}

\hypertarget{datum-5}{%
\subsection{Datum}\label{datum-5}}

\begin{itemize}
\tightlist
\item
  18.4.-22.4
\end{itemize}

\hypertarget{hinweise-2}{%
\subsection{Hinweise}\label{hinweise-2}}

\begin{itemize}
\tightlist
\item
  In dieser Woche fällt die Vorlesung aus.
\end{itemize}

\hypertarget{logistische-regression}{%
\section{Logistische Regression}\label{logistische-regression}}

\hypertarget{datum-6}{%
\subsection{Datum}\label{datum-6}}

\begin{itemize}
\tightlist
\item
  25.4.-29.4.
\end{itemize}

\hypertarget{literatur-6}{%
\subsection{Literatur}\label{literatur-6}}

\begin{itemize}
\tightlist
\item
  Rhys, Kap. 4
\end{itemize}

\hypertarget{naive-bayes}{%
\section{Naive Bayes}\label{naive-bayes}}

\hypertarget{datum-7}{%
\subsection{Datum}\label{datum-7}}

\begin{itemize}
\tightlist
\item
  2.4.-6.5.
\end{itemize}

\hypertarget{literatur-7}{%
\subsection{Literatur}\label{literatur-7}}

\begin{itemize}
\tightlist
\item
  Rhys, Kap. 6
\end{itemize}

\hypertarget{entscheidungsbuxe4ume}{%
\section{Entscheidungsbäume}\label{entscheidungsbuxe4ume}}

\hypertarget{datum-8}{%
\subsection{Datum}\label{datum-8}}

\begin{itemize}
\tightlist
\item
  9.5.-13.5.
\end{itemize}

\hypertarget{literatur-8}{%
\subsection{Literatur}\label{literatur-8}}

\begin{itemize}
\tightlist
\item
  Rhys, Kap. 7
\end{itemize}

\hypertarget{zufallswuxe4lder}{%
\section{Zufallswälder}\label{zufallswuxe4lder}}

\hypertarget{datum-9}{%
\subsection{Datum}\label{datum-9}}

\begin{itemize}
\tightlist
\item
  16.5.-20.5.
\end{itemize}

\hypertarget{literatur-9}{%
\subsection{Literatur}\label{literatur-9}}

\begin{itemize}
\tightlist
\item
  Rhys, Kap. 8
\end{itemize}

\hypertarget{fallstudie}{%
\section{Fallstudie}\label{fallstudie}}

\hypertarget{datum-10}{%
\subsection{Datum}\label{datum-10}}

\begin{itemize}
\tightlist
\item
  23.5.-27.5.
\end{itemize}

\hypertarget{literatur-10}{%
\subsection{Literatur}\label{literatur-10}}

\begin{itemize}
\tightlist
\item
  Rhys, Kap.9
\end{itemize}

\hypertarget{hinweise-3}{%
\subsection{Hinweise}\label{hinweise-3}}

\begin{itemize}
\tightlist
\item
  Nächste Woche ist BlocKalenderwocheoche; es findet kein regulärer Unterricht statt.
\item
  Diese Woche fällt die Übung aus.
\end{itemize}

\hypertarget{wiederholung-1}{%
\section{Wiederholung}\label{wiederholung-1}}

\hypertarget{datum-11}{%
\subsection{Datum}\label{datum-11}}

\begin{itemize}
\tightlist
\item
  6.6.-10.6.
\end{itemize}

\hypertarget{hinweise-4}{%
\subsection{Hinweise}\label{hinweise-4}}

\begin{itemize}
\tightlist
\item
  In dieser Woche fällt die Vorlesung aus.
\end{itemize}

\hypertarget{gam}{%
\section{GAM}\label{gam}}

\hypertarget{datum-12}{%
\subsection{Datum}\label{datum-12}}

\begin{itemize}
\tightlist
\item
  13.6.-17.6.
\end{itemize}

\hypertarget{literatur-11}{%
\subsection{Literatur}\label{literatur-11}}

\begin{itemize}
\tightlist
\item
  Rhys, Kap. 10
\end{itemize}

\hypertarget{lasso-und-co}{%
\section{Lasso und Co}\label{lasso-und-co}}

\hypertarget{datum-13}{%
\subsection{Datum}\label{datum-13}}

\begin{itemize}
\tightlist
\item
  20.6.-24.6.
\end{itemize}

\hypertarget{literatur-12}{%
\subsection{Literatur}\label{literatur-12}}

\begin{itemize}
\tightlist
\item
  Rhys, Kap. 11
\end{itemize}

\hypertarget{vertiefung}{%
\section{Vertiefung}\label{vertiefung}}

\hypertarget{datum-14}{%
\subsection{Datum}\label{datum-14}}

\begin{itemize}
\tightlist
\item
  27.6.-1.7.
\end{itemize}

\hypertarget{literatur-13}{%
\subsection{Literatur}\label{literatur-13}}

\begin{itemize}
\tightlist
\item
  Rhys, Kap. 12
\end{itemize}

\hypertarget{hinweise-5}{%
\subsection{Hinweise}\label{hinweise-5}}

\begin{itemize}
\tightlist
\item
  Nach dieser Woche endet der Unterricht.
\end{itemize}

\hypertarget{modulzeitplan}{%
\chapter{Modulzeitplan}\label{modulzeitplan}}

\begin{longtable}[]{@{}llll@{}}
\toprule
Nr. & Kalenderwoche & Datum & Thema \\
\midrule
\endhead
1 & 11 & 14.-18.3.22 & Grundkonzepte \\
2 & 12 & 21.3.-25.3. & tidyverse, 2. Blick \\
3 & 13 & 28.3.-1.4. & tidymodels \\
4 & 14 & 4.4.-8.4. & kNN \\
5 & 15 & 11.4.-15.4. & Statistisches Lernen \\
6 & 16 & 18.4.-22.4 & Wiederholung \\
7 & 17 & 25.4.-29.4. & Logistische Regression \\
8 & 18 & 2.4.-6.5. & Naive Bayes \\
9 & 19 & 9.5.-13.5. & Entscheidungsbäume \\
10 & 20 & 16.5.-20.5. & Zufallswälder \\
11 & 21 & 23.5.-27.5. & Fallstudie \\
12 & 23 & 6.6.-10.6. & Wiederholung \\
13 & 24 & 13.6.-17.6. & GAM \\
14 & 25 & 20.6.-24.6. & Lasso und Co \\
15 & 26 & 27.6.-1.7. & Vertiefung \\
\bottomrule
\end{longtable}

  \bibliography{book.bib,packages.bib}

\end{document}
